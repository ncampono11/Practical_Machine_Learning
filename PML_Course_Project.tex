% Options for packages loaded elsewhere
\PassOptionsToPackage{unicode}{hyperref}
\PassOptionsToPackage{hyphens}{url}
%
\documentclass[
]{article}
\usepackage{lmodern}
\usepackage{amssymb,amsmath}
\usepackage{ifxetex,ifluatex}
\ifnum 0\ifxetex 1\fi\ifluatex 1\fi=0 % if pdftex
  \usepackage[T1]{fontenc}
  \usepackage[utf8]{inputenc}
  \usepackage{textcomp} % provide euro and other symbols
\else % if luatex or xetex
  \usepackage{unicode-math}
  \defaultfontfeatures{Scale=MatchLowercase}
  \defaultfontfeatures[\rmfamily]{Ligatures=TeX,Scale=1}
\fi
% Use upquote if available, for straight quotes in verbatim environments
\IfFileExists{upquote.sty}{\usepackage{upquote}}{}
\IfFileExists{microtype.sty}{% use microtype if available
  \usepackage[]{microtype}
  \UseMicrotypeSet[protrusion]{basicmath} % disable protrusion for tt fonts
}{}
\makeatletter
\@ifundefined{KOMAClassName}{% if non-KOMA class
  \IfFileExists{parskip.sty}{%
    \usepackage{parskip}
  }{% else
    \setlength{\parindent}{0pt}
    \setlength{\parskip}{6pt plus 2pt minus 1pt}}
}{% if KOMA class
  \KOMAoptions{parskip=half}}
\makeatother
\usepackage{xcolor}
\IfFileExists{xurl.sty}{\usepackage{xurl}}{} % add URL line breaks if available
\IfFileExists{bookmark.sty}{\usepackage{bookmark}}{\usepackage{hyperref}}
\hypersetup{
  pdftitle={PML Course Project},
  pdfauthor={Nick Campono},
  hidelinks,
  pdfcreator={LaTeX via pandoc}}
\urlstyle{same} % disable monospaced font for URLs
\usepackage[margin=1in]{geometry}
\usepackage{color}
\usepackage{fancyvrb}
\newcommand{\VerbBar}{|}
\newcommand{\VERB}{\Verb[commandchars=\\\{\}]}
\DefineVerbatimEnvironment{Highlighting}{Verbatim}{commandchars=\\\{\}}
% Add ',fontsize=\small' for more characters per line
\usepackage{framed}
\definecolor{shadecolor}{RGB}{248,248,248}
\newenvironment{Shaded}{\begin{snugshade}}{\end{snugshade}}
\newcommand{\AlertTok}[1]{\textcolor[rgb]{0.94,0.16,0.16}{#1}}
\newcommand{\AnnotationTok}[1]{\textcolor[rgb]{0.56,0.35,0.01}{\textbf{\textit{#1}}}}
\newcommand{\AttributeTok}[1]{\textcolor[rgb]{0.77,0.63,0.00}{#1}}
\newcommand{\BaseNTok}[1]{\textcolor[rgb]{0.00,0.00,0.81}{#1}}
\newcommand{\BuiltInTok}[1]{#1}
\newcommand{\CharTok}[1]{\textcolor[rgb]{0.31,0.60,0.02}{#1}}
\newcommand{\CommentTok}[1]{\textcolor[rgb]{0.56,0.35,0.01}{\textit{#1}}}
\newcommand{\CommentVarTok}[1]{\textcolor[rgb]{0.56,0.35,0.01}{\textbf{\textit{#1}}}}
\newcommand{\ConstantTok}[1]{\textcolor[rgb]{0.00,0.00,0.00}{#1}}
\newcommand{\ControlFlowTok}[1]{\textcolor[rgb]{0.13,0.29,0.53}{\textbf{#1}}}
\newcommand{\DataTypeTok}[1]{\textcolor[rgb]{0.13,0.29,0.53}{#1}}
\newcommand{\DecValTok}[1]{\textcolor[rgb]{0.00,0.00,0.81}{#1}}
\newcommand{\DocumentationTok}[1]{\textcolor[rgb]{0.56,0.35,0.01}{\textbf{\textit{#1}}}}
\newcommand{\ErrorTok}[1]{\textcolor[rgb]{0.64,0.00,0.00}{\textbf{#1}}}
\newcommand{\ExtensionTok}[1]{#1}
\newcommand{\FloatTok}[1]{\textcolor[rgb]{0.00,0.00,0.81}{#1}}
\newcommand{\FunctionTok}[1]{\textcolor[rgb]{0.00,0.00,0.00}{#1}}
\newcommand{\ImportTok}[1]{#1}
\newcommand{\InformationTok}[1]{\textcolor[rgb]{0.56,0.35,0.01}{\textbf{\textit{#1}}}}
\newcommand{\KeywordTok}[1]{\textcolor[rgb]{0.13,0.29,0.53}{\textbf{#1}}}
\newcommand{\NormalTok}[1]{#1}
\newcommand{\OperatorTok}[1]{\textcolor[rgb]{0.81,0.36,0.00}{\textbf{#1}}}
\newcommand{\OtherTok}[1]{\textcolor[rgb]{0.56,0.35,0.01}{#1}}
\newcommand{\PreprocessorTok}[1]{\textcolor[rgb]{0.56,0.35,0.01}{\textit{#1}}}
\newcommand{\RegionMarkerTok}[1]{#1}
\newcommand{\SpecialCharTok}[1]{\textcolor[rgb]{0.00,0.00,0.00}{#1}}
\newcommand{\SpecialStringTok}[1]{\textcolor[rgb]{0.31,0.60,0.02}{#1}}
\newcommand{\StringTok}[1]{\textcolor[rgb]{0.31,0.60,0.02}{#1}}
\newcommand{\VariableTok}[1]{\textcolor[rgb]{0.00,0.00,0.00}{#1}}
\newcommand{\VerbatimStringTok}[1]{\textcolor[rgb]{0.31,0.60,0.02}{#1}}
\newcommand{\WarningTok}[1]{\textcolor[rgb]{0.56,0.35,0.01}{\textbf{\textit{#1}}}}
\usepackage{graphicx,grffile}
\makeatletter
\def\maxwidth{\ifdim\Gin@nat@width>\linewidth\linewidth\else\Gin@nat@width\fi}
\def\maxheight{\ifdim\Gin@nat@height>\textheight\textheight\else\Gin@nat@height\fi}
\makeatother
% Scale images if necessary, so that they will not overflow the page
% margins by default, and it is still possible to overwrite the defaults
% using explicit options in \includegraphics[width, height, ...]{}
\setkeys{Gin}{width=\maxwidth,height=\maxheight,keepaspectratio}
% Set default figure placement to htbp
\makeatletter
\def\fps@figure{htbp}
\makeatother
\setlength{\emergencystretch}{3em} % prevent overfull lines
\providecommand{\tightlist}{%
  \setlength{\itemsep}{0pt}\setlength{\parskip}{0pt}}
\setcounter{secnumdepth}{-\maxdimen} % remove section numbering

\title{PML Course Project}
\author{Nick Campono}
\date{7/20/2020}

\begin{document}
\maketitle

One thing that people regularly do is quantify how much of a particular
activity they do, but they rarely quantify how well they do it.

Our for this project will be to use data from accelerometers on the
belt, forearm, arm, and dumbell of 6 participants to predict the mannar
in which someone performs their exercise.

\hypertarget{load-and-clean-data}{%
\subsection{Load and Clean data}\label{load-and-clean-data}}

\begin{Shaded}
\begin{Highlighting}[]
\KeywordTok{setwd}\NormalTok{(}\StringTok{'C:/Users/ncampono/Desktop/R/Courses/Practical Machine Learning/Practical_Machine_Learning'}\NormalTok{)}
\NormalTok{trainingPML =}\StringTok{ }\KeywordTok{read.csv}\NormalTok{(}\StringTok{'pml-training.csv'}\NormalTok{, }\DataTypeTok{na.strings=}\KeywordTok{c}\NormalTok{(}\StringTok{"NA"}\NormalTok{,}\StringTok{"#DIV/0!"}\NormalTok{,}\StringTok{""}\NormalTok{))}
\NormalTok{testingPML =}\StringTok{ }\KeywordTok{read.csv}\NormalTok{(}\StringTok{'pml-testing.csv'}\NormalTok{, }\DataTypeTok{na.strings=}\KeywordTok{c}\NormalTok{(}\StringTok{"NA"}\NormalTok{,}\StringTok{"#DIV/0!"}\NormalTok{,}\StringTok{""}\NormalTok{))}

\CommentTok{# delete variables that we can't use for prediction}
\NormalTok{trainingPML <-}\StringTok{ }\NormalTok{trainingPML[,}\OperatorTok{-}\KeywordTok{c}\NormalTok{(}\DecValTok{1}\OperatorTok{:}\DecValTok{7}\NormalTok{)]}
\NormalTok{testingPML <-}\StringTok{ }\NormalTok{testingPML[,}\OperatorTok{-}\KeywordTok{c}\NormalTok{(}\DecValTok{1}\OperatorTok{:}\DecValTok{7}\NormalTok{)]}


\CommentTok{# delete columns with all missing data}
\NormalTok{trainingPML<-trainingPML[,}\KeywordTok{colSums}\NormalTok{(}\KeywordTok{is.na}\NormalTok{(trainingPML)) }\OperatorTok{==}\StringTok{ }\DecValTok{0}\NormalTok{]}
\NormalTok{testingPML <-testingPML[,}\KeywordTok{colSums}\NormalTok{(}\KeywordTok{is.na}\NormalTok{(testingPML)) }\OperatorTok{==}\StringTok{ }\DecValTok{0}\NormalTok{]}
\end{Highlighting}
\end{Shaded}

\hypertarget{create-training-and-test-datasets-for-modeling}{%
\subsection{Create Training and Test datasets for
modeling}\label{create-training-and-test-datasets-for-modeling}}

\begin{Shaded}
\begin{Highlighting}[]
\KeywordTok{set.seed}\NormalTok{(}\DecValTok{1234}\NormalTok{)}
\NormalTok{inTrain <-}\StringTok{ }\KeywordTok{createDataPartition}\NormalTok{(}\DataTypeTok{y=}\NormalTok{trainingPML}\OperatorTok{$}\NormalTok{classe, }\DataTypeTok{p=}\FloatTok{0.7}\NormalTok{, }\DataTypeTok{list=}\OtherTok{FALSE}\NormalTok{)}
\NormalTok{training <-}\StringTok{ }\NormalTok{trainingPML[inTrain,]}
\NormalTok{testing <-}\StringTok{ }\NormalTok{trainingPML[}\OperatorTok{-}\NormalTok{inTrain,]}

\KeywordTok{dim}\NormalTok{(training)}
\end{Highlighting}
\end{Shaded}

\begin{verbatim}
## [1] 13737    53
\end{verbatim}

\begin{Shaded}
\begin{Highlighting}[]
\KeywordTok{dim}\NormalTok{(testing)}
\end{Highlighting}
\end{Shaded}

\begin{verbatim}
## [1] 5885   53
\end{verbatim}

the data plot the frequency of the classe variable to see what the
frequency in the training set is

\begin{Shaded}
\begin{Highlighting}[]
\KeywordTok{plot}\NormalTok{(trainingPML}\OperatorTok{$}\NormalTok{classe, }\DataTypeTok{col =} \StringTok{'#00004d'}\NormalTok{, }\DataTypeTok{main=}\StringTok{'Variable Classe Frequency'}\NormalTok{, }\DataTypeTok{xlab=}\StringTok{'classe'}\NormalTok{, }\DataTypeTok{ylab=}\StringTok{'Frequency'}\NormalTok{)}
\end{Highlighting}
\end{Shaded}

\includegraphics{PML_Course_Project_files/figure-latex/unnamed-chunk-4-1.pdf}

\hypertarget{prediction-model-1-decision-tree}{%
\subsection{Prediction Model 1: Decision
Tree}\label{prediction-model-1-decision-tree}}

\begin{Shaded}
\begin{Highlighting}[]
\NormalTok{model.DT <-}\StringTok{ }\KeywordTok{rpart}\NormalTok{(classe }\OperatorTok{~}\StringTok{ }\NormalTok{., }\DataTypeTok{data=}\NormalTok{training, }\DataTypeTok{method=}\StringTok{"class"}\NormalTok{)}

\NormalTok{pred.DT<-}\StringTok{ }\KeywordTok{predict}\NormalTok{(model.DT, testing, }\DataTypeTok{type =} \StringTok{'class'}\NormalTok{)}

\CommentTok{# decision tree plot}

\KeywordTok{rpart.plot}\NormalTok{(model.DT, }\DataTypeTok{main=}\StringTok{"Classification Tree"}\NormalTok{, }\DataTypeTok{extra=}\DecValTok{102}\NormalTok{, }\DataTypeTok{under=}\OtherTok{TRUE}\NormalTok{, }\DataTypeTok{faclen=}\DecValTok{0}\NormalTok{)}
\end{Highlighting}
\end{Shaded}

\includegraphics{PML_Course_Project_files/figure-latex/unnamed-chunk-5-1.pdf}

\begin{Shaded}
\begin{Highlighting}[]
\CommentTok{# confusion matrix}
\KeywordTok{confusionMatrix}\NormalTok{(pred.DT, testing}\OperatorTok{$}\NormalTok{classe)}
\end{Highlighting}
\end{Shaded}

\begin{verbatim}
## Confusion Matrix and Statistics
## 
##           Reference
## Prediction    A    B    C    D    E
##          A 1522  167   12   49   13
##          B   58  706  100   79   96
##          C   47  109  819  148  139
##          D   25   94   67  609   52
##          E   22   63   28   79  782
## 
## Overall Statistics
##                                           
##                Accuracy : 0.7541          
##                  95% CI : (0.7429, 0.7651)
##     No Information Rate : 0.2845          
##     P-Value [Acc > NIR] : < 2.2e-16       
##                                           
##                   Kappa : 0.6885          
##                                           
##  Mcnemar's Test P-Value : < 2.2e-16       
## 
## Statistics by Class:
## 
##                      Class: A Class: B Class: C Class: D Class: E
## Sensitivity            0.9092   0.6198   0.7982   0.6317   0.7227
## Specificity            0.9428   0.9298   0.9088   0.9516   0.9600
## Pos Pred Value         0.8633   0.6795   0.6490   0.7190   0.8029
## Neg Pred Value         0.9631   0.9106   0.9552   0.9295   0.9389
## Prevalence             0.2845   0.1935   0.1743   0.1638   0.1839
## Detection Rate         0.2586   0.1200   0.1392   0.1035   0.1329
## Detection Prevalence   0.2996   0.1766   0.2144   0.1439   0.1655
## Balanced Accuracy      0.9260   0.7748   0.8535   0.7917   0.8414
\end{verbatim}

Accuracy for the Decision Tree is 68.79\%

\hypertarget{prediction-model-2-random-forest}{%
\subsection{Prediction Model 2: Random
Forest}\label{prediction-model-2-random-forest}}

\begin{Shaded}
\begin{Highlighting}[]
\NormalTok{model.RF<-}\StringTok{ }\KeywordTok{randomForest}\NormalTok{(classe }\OperatorTok{~}\NormalTok{. , }\DataTypeTok{data =}\NormalTok{ training, }\DataTypeTok{method =} \StringTok{'class'}\NormalTok{)}

\NormalTok{pred.RF <-}\StringTok{ }\KeywordTok{predict}\NormalTok{(model.RF, training, }\DataTypeTok{type =} \StringTok{"class"}\NormalTok{)}

\CommentTok{# confusion matrix}
\KeywordTok{confusionMatrix}\NormalTok{(pred.RF, training}\OperatorTok{$}\NormalTok{classe)}
\end{Highlighting}
\end{Shaded}

\begin{verbatim}
## Confusion Matrix and Statistics
## 
##           Reference
## Prediction    A    B    C    D    E
##          A 3906    0    0    0    0
##          B    0 2658    0    0    0
##          C    0    0 2396    0    0
##          D    0    0    0 2252    0
##          E    0    0    0    0 2525
## 
## Overall Statistics
##                                      
##                Accuracy : 1          
##                  95% CI : (0.9997, 1)
##     No Information Rate : 0.2843     
##     P-Value [Acc > NIR] : < 2.2e-16  
##                                      
##                   Kappa : 1          
##                                      
##  Mcnemar's Test P-Value : NA         
## 
## Statistics by Class:
## 
##                      Class: A Class: B Class: C Class: D Class: E
## Sensitivity            1.0000   1.0000   1.0000   1.0000   1.0000
## Specificity            1.0000   1.0000   1.0000   1.0000   1.0000
## Pos Pred Value         1.0000   1.0000   1.0000   1.0000   1.0000
## Neg Pred Value         1.0000   1.0000   1.0000   1.0000   1.0000
## Prevalence             0.2843   0.1935   0.1744   0.1639   0.1838
## Detection Rate         0.2843   0.1935   0.1744   0.1639   0.1838
## Detection Prevalence   0.2843   0.1935   0.1744   0.1639   0.1838
## Balanced Accuracy      1.0000   1.0000   1.0000   1.0000   1.0000
\end{verbatim}

Accuracy for the Random Forest is 100\%. Random Forest algorithm
performed better than Decision Trees. We will use the Random Forest
prediction model for modeling the testing dataset.

\hypertarget{predict-testing-data}{%
\subsection{Predict Testing Data}\label{predict-testing-data}}

predict the testing data set using the random forest

\begin{Shaded}
\begin{Highlighting}[]
\NormalTok{pred.FINAL <-}\StringTok{ }\KeywordTok{predict}\NormalTok{(model.RF, testingPML, }\DataTypeTok{type=}\StringTok{"class"}\NormalTok{)}
\NormalTok{pred.FINAL}
\end{Highlighting}
\end{Shaded}

\begin{verbatim}
##  1  2  3  4  5  6  7  8  9 10 11 12 13 14 15 16 17 18 19 20 
##  B  A  B  A  A  E  D  B  A  A  B  C  B  A  E  E  A  B  B  B 
## Levels: A B C D E
\end{verbatim}

\hypertarget{discussion}{%
\subsection{Discussion}\label{discussion}}

For this project 19622 observations from weight lifting were used to
predict correct body movement during. We looked at two models Random
Forest and Decision Trees. The Decision Tree model wasn't as accurate as
the Random Forest model at predicting the classe. The Random Forest
model had a 100\% accuracy rate for the testing dataset. We used this
model to predict the test data and correctly identified all 20 classe's.

\end{document}
